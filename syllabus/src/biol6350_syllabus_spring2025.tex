\documentclass[12pt, notitlepage]{article}   	% use "amsart" instead of "article" for AMSLaTeX format
\usepackage{geometry}                		% See geometry.pdf to learn the layout options. There are lots.
\geometry{a4paper}                   		% ... or a4paper or a5paper or ... 
%\geometry{landscape}                		% Activate for rotated page geometry
\usepackage[parfill]{parskip}    		% Activate to begin paragraphs with an empty line rather than an indent
\usepackage{graphicx}				% Use pdf, png, jpg, or eps§ with pdflatex; use eps in DVI mode
								% TeX will automatically convert eps --> pdf in pdflatex

\usepackage{hyperref}
		
%SetFonts

\usepackage[T1]{fontenc}
\usepackage[utf8]{inputenc}

\usepackage{tgbonum}

%SetFonts

\title{
	\textbf{
		BIOL 6350
	} \\
	\large Advanced Physiological Plant Ecology \\
	\large Spring 2025
}

\date{\vspace{-5ex}}

\begin{document}

{\fontfamily{phv}\selectfont %select helvetica (code = phv)

\maketitle

\section{Course Description}
Students in this course will learn the fundamentals of plant physiology through 
an ecological lens. The course will focus on plant responses to environmental conditions 
across multiple spatial and temporal scales. The course will cover plant, water, carbon, 
and nutrient relations in natural and managed systems across multiple ecological scales. 
Students will be evaluated on their 
ability to discuss and disseminate ecophysiological topics.

\subsection{Class Time and Location}
Tuesdays and Thursdays 9:30-10:50

Biology Building (BIOL) Room 102-103.

\subsection{Instructor}
Dr. Nick Smith \par
Experimental Sciences Building II (ESBII) Room 402D \par
806-834-7363 \par
nick.smith@ttu.edu \par

\subsection{Office Hours}
Tuesdays 14:00-14:50

\subsection{Recommended Texts}
Plant Physiological Ecology (2nd Edition; 2008) by Lambers, Chapin, and Pons \par
The book can be accessed from Springer here: 
\url{https://www.springer.com/us/book/9780387783406}. Click on "Access this title on 
SpringerLink." It can also be accessed through the TTU library. \par
Plant Physiology and Development (6th Edition) by Taiz, Ziegler, Moller, and Murphy

\section{Mode of Instruction}
All instruction will be done face-to-face unless the university directs classes be 
taught online.

\section{Course Materials}
All course materials, including lecture slides, readings, activities, and code 
will be posted to a GitHub repository for the course.
The primary repository address is
\url{https://github.com/SmithEcophysLab/biol43506350_spring2025}.

\section{Learning Objective}
This course will broadly focus on understanding the role that plant physiological 
processes play in driving ecological responses across multiple scales from the individual 
to the globe. Class activities will be based on discussion and dissemination of ideas, 
including classic and recent scientific literature. 
Topics will be flexible and modified to match student interests where possible.

\section{Attendance Policy}
Attendance is strongly recommended. 
Much of the graded content will be completed in class and students will not be permitted
to complete this material outside of class. See course assessments below for details.
Makeups will not be granted.

\section{Course Assessment}
\subsection{\textit{Participation and Engagement}}
Being an active and engaged participant in the class will benefit your understanding
of material as well as your peers'. Examples include asking questions, providing feedback,
and facilitating discussion. Participation and engagement of each student will be monitored
during each class period.

\subsection{\textit{Mini-quizzes}}
Short “quizzes” will be given in class each week (typically on Thursdays). 
These quizzes will be used to stimulate discussion and to assess how well 
prior concepts were understood by the class.

\subsection{\textit{Classical literature feedback}}
Each week students will be required to read a “classic” article on the current weeks’ 
topic and produce a short summary as well as two questions that arose during their 
reading of the article. Students are encouraged to bring up these questions during the
Tuesday class discussions.

\subsection{\textit{Recent literature article lead}}
Each student will be required to lead one Thursday discussion on recent literature. 
This will involve introducing the article 
and leading a jigsaw-style discussion related to the article. Students must read some of the cited
literature integral to the study 
in order to answer relevant questions brought forth during the discussion.
The article will be chosen by Dr. Smith, unless a different arrangement is made.
Discussion leads will be done in groups of 1-3 students.

\subsection{\textit{Recent literature article feedback}}
Students not leading the current week’s 
discussion will be required to produce a summary and 
develop two questions based on each week’s article.

\subsection{\textit{Literature Review}}
The primary semester project will be to produce a literature review on a topic 
of the student's choice.
Broadly, the review should address a question or problem related 
to plant ecophysiology and review the current state of knowledge on the topic.
The review should be forward thinking, in that it forms the
basis for understanding plant physiological processes moving forward.
The review should be novel in that it should not be similar to previously published
review papers.

Students will first develop a written proposal for their literature review and present 
their idea to the class. The class and instructors will provide feedback. Students will then produce and present 
their review to the class at the end of the semester. 

This project will be done individually. Students are encouraged to receive help and guidance 
from the instructors as well as the class at large. 

The literature review will be assessed for completeness, breadth, originality, and presentation.
Students must have their project OKed by the instructor after the proposal and prior to
beginning the final project.

The grading of this review will differ from that for the undergraduate section. Specifically,
for full points on the breadth and originality portion (20\% of written grade), the question addressed must be completely novel and
not something that has been reviewed in the published literature. The review
should produce novel findings, questions, and hypotheses. For full points on the scientific rigor portion (30\% of written grade),
the content of the review should be of suitable quality for partial inclusion
in the introduction of a manuscript or proposal or be submittable for
publication on its own. For full points on the oral presentation portion (5\% of total oral grade), 
the presentation should be of suitable quality that it could be presented
at a national conference in the field of plant science (Botanical Society of America) and/or ecology
(Ecological Society of America).

\section{Grading}
Participation and Engagement: 15\% \par
Mini-quizzes: 10\% \par
Classical literature feedback: 5\% \par
Recent literature lead: 15\% \par
Recent literature feedback: 5\% \par
Review idea proposal: 10\% \par
Review idea feedback: 5\% \par
Final review presentation: 10\% \par
Final review: 25\% \par

Grades will be made available on Blackboard. 
All grades posted at the end of the course will be final.
Please contact Dr. Smith if you feel your grade has been calculated incorrectly.

\section{Grading Scale}
A: $\geq$ 90\% \par
B: 80 – 90\% \par
C: 70 – 80\% \par
D: 60 – 70\% \par
F: $\leq$ 59.9\% \par

\section{Missing In-class Activities}
Students will be required to be in class to receive in-class activity points. 
Please contact Dr. Smith if you plan to miss class for a university function 
\textit{prior to class}. If class is missed due to an illness, 
please let Dr. Smith know as soon as possible. Documentation will need to be provided
in order to be able to make up any missed work.

\section{Special Considerations}
Texas Tech Policies Concerning Academic Honesty, Special Accommodations for 
Students with Disabilities, Student Absences for Observance of Religious Holy Days, 
Accommodations for Pregnant Students, and other policies may be found at this link: 
\url{https://www.depts.ttu.edu/tlpdc/RequiredSyllabusStatements.php}.

\subsection{AI Use}
The use of generative AI tools (such as ChatGPT) is strictly prohibited in this course for any purpose.
Information gathered from AI cannot be used even with appropriate citation. Submission of AI-generated
content (i.e., information, text, or images) as your own work is a violation of academic integrity and may
result in referral to the Office of Student Conduct. Please contact your instructor if you have questions
regarding this course policy.

\section{Plagiarism Statement}
Texas Tech University expects students to “understand the principles of academic integrity 
and abide by them in all class and/or course work at the University” (OP 34.12.5). 
Plagiarism is a form of academic misconduct that involves (1) the representation of words, 
ideas, illustrations, structure, computer code, other expression, or media of another as 
one's own and/or failing to properly cite direct, paraphrased, or summarized materials; 
or (2) self-plagiarism, which involves the submission of the same academic work more than 
once without the prior permission of the instructor and/or failure to correctly cite 
previous work written by the same student. Please review Section B of the TTU 
Student Handbook for more information related to other forms of academic misconduct, 
and contact your instructor if you have questions about plagiarism or other 
academic concerns in your courses. To learn more about the importance of 
academic integrity and practical tips for avoiding plagiarism, explore the 
resources provided by the TTU Library and the School of Law.

\newpage

\section*{Schedule of Topics by Week; pages refer to pages in Lambers book}
13/01/23 - Introductions, semester planning, and goals \par
20/01/23 – Physiology’s role in ecology (pp. 1-8) \par
27/01/23 – Key physiological processes: 
photosynthesis, respiration, transpiration, translocation (pp. 11-203) \par
03/02/23 – Light (pp. 26-47, 237-238) \par
10/02/23 – Temperature (pp. 60-63, 127-129, 239-244) \par
17/02/23 – Water (53-57, 163-217) \par
24/02/23 – CO2 (pp. 87-90) \par
03/03/23 – Nutrients (pp. 58-59, 225-310) \par
10/03/23 – \textbf{Literature review proposal presentations} \par
17/03/23 – NO CLASS \par
24/03/23 – Growth and allocation (pp. 321-367), Life cycles, ontogeny, and phenology (pp. 375-398) \par
31/03/23 – Competition (pp. 505-518) \par
07/04/23 – Scaling (pp. 555-569) \par
14/04/23 – \textbf{Literature review presentations} \par
21/04/23 – \textbf{Literature review presentations} \par
28/04/23 – NO CLASS \par
05/05/23 – NO CLASS \par

\section*{General Weekly Schedule}
Generally, each Tuesday will consist of a lecture by Dr. Smith followed by a discussion
of a classical literature article. Students will turn in their classical literature
feedback at the end of Tuesday's lecture. Thursdays will generally begin with an in-class
mini-quiz and discussion. 
This will be followed by a discussion of a recent literature article and
(time permitting) an in-class activity.

} %end font selection

\end{document} 